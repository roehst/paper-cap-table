
Before stating the properties to be check, a set of functions to query the number of shares in different contexts and also to trace the lineage of \glspl{security} need to be defined (Listing \ref{lst:tt-alloy-functions}). 

%\subsubsection{Queries and functions.}

%Before we can state the properties we intend to check, we need to define a few functions to query the number of shares in different contexts and also to trace the lineage of \glspl{security}. 

\begin{listing}[!h]
 \begin{multicols}{2}
	\begin{minted}[fontsize=\footnotesize]
     {alloy}
fun lineage[sec : Security] : set Security {
    sec.*(Security <: parent)
}

fun depth[sec : Security] : Int {
    #lineage[sec]
}

fun aliveSecurities : set Security {
    { sec : Security | no sec.use }
}
fun deadSecurities : set Security {
    { sec : Security | some sec.use }
}

fun issuedShares : Int {
    sum iss : Issuance | iss.shares
}

fun cancelledShares : Int {
    sum can : Cancellation | can.shares
}

fun transferredShares : Int {
    sum xfer : Transfer | xfer.shares
}

fun aliveShares : Int {
    sum sec : aliveSecurities | sec.shares
}

fun deadShares : Int {
    sum sec : deadSecurities | sec.shares
}
\end{minted}
\end{multicols}
	\caption{Alloy Functions}
\label{lst:tt-alloy-functions}
\end{listing}

The lineage of \glspl{security} is defined using the \verb|*| transitive closure operator, showing a succinct definition embedded in Alloy.
Another group of functions supports accounting identities (related to Shares).
%(Listing \ref{lst:tt-alloy-functions-accounting}).