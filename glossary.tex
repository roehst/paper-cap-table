\newglossaryentry{security}{
    name=security,
    description={A security is a financial asset that can be bought and sold. Stocks, options and debt notes are all securities. Every security has an Issuer. A loan from a bank is not a security, because the bank can not generally sell the loan to another bank.},
    plural=securities
}

\newglossaryentry{asset}{
    name=asset,
    description={An asset is something that can eventually generate cashflows. Because not all future cashflows are known with certainty, the value of an asset must be discounted to reflect the risk that those cashflows do not meet expectations.},
    plural=assets
}

\newglossaryentry{issuer}{
    name=issuer,
    description={Companies are issuers of securities.},
    plural=issuers
}

\newglossaryentry{stock}{
    name=stock,
    description={Stocks are securities that represent ownership in a company. Stock are typically issued as shares (e.g. 100 shares of Apple stock). Shares are fractions of a company's total ownership.}
    plural=stocks
}

\newglossaryentry{debt}{
    name=debt,
    description={Debt is a loan that must be repaid. Companies might raise funds via equity issuances or debt issuances. Debt is issued as security in terms of the amount that was loaned, the interest rate, and the maturity date. Debt is safer than equity, and must be repaid before equity holders can receive any cashflows.},
    plural=debt
}

\newglossaryentry{convertible-debt}{
    name={convertible debt},
    description={In startup financing, it is typical to encounter convertible debt. Convertible debt is a loan that can be converted into equity at a later date. The conversion is typically triggered by a future financing round. The conversion price is typically set at a discount to the price of the future financing round. Since debt is safer, convertible debt has lower investment risk. Conversion is typically at the option of the holder.},
    plural={convertible debt}
}

\newglossaryentry{stock-option}{
    name={stock option},
    description={Stock options are securities that give the right for their holder to purchase stock at a predefined price (the strike price) at a predefined date (the maturity date). The value of a stock option is the different between the market price for the stock and the strike price. Stock options are typically issued to employees as part of their compensation package.},
    plural={stock options}

}

\newglossaryentry{startup-company}{
    name={startup company},
    description={A startup company is a new company that is searching for a business model as it grows. Startup companies are typically funded in stages and by specialized venture capital investors such as individual (angel) investors and funds. Startup companies usually aim for high growth and high returns, by choosing projects with higher risk.},
    plural={startup companies}
}

\newglossaryentry{capitalization-table}{
    name={capitalization table},
    description={A capitalization table is a table that lists all the securities issued by a company. The capitalization table lists the number of shares issued, the type of security, the price per share, and the date of issuance. The capitalization table is used to calculate the ownership of each shareholder.},
    plural={capitalization tables}
}

\newglossaryentry{vesting-period}{
    name={vesting period},
    description={A vesting period is a period of time during which an employee must remain employed in order to receive the full value of their stock options. Vesting periods are typically 4 years, with a 1 year cliff period. Only vested stock options can be execised.},
    plural={vesting periods}
}


\newglossaryentry{cliff-period}{
    name={cliff period},
    description={The cliff period is the period of time before which no stock options are vested. After the cliff period, stock options are typically vested monthly.},
    plural={cliff periods}
}

\newglossaryentry{strike-price}{
    name={strike price},
    description={The strike price is the price at which a stock option can be exercised. The strike price is typically set at the market price of the stock at the time of issuance, but may be futher discounted to incentivize employees.},
    plural={strike prices}
}

\newglossaryentry{common-stock}{
    name={common stock},
    description={Stock that holds no special rights beyond a share in profits. Common stock is the most common type of stock.},
    plural={common stock}
}

\newglossaryentry{preferred-stock}{
    name={preferred stock},
    description={Preferred stock is stock that holds special rights. As an example of a special right, preferred stock might have a guaranteed dividend payment but less voting rights.},
    plural={preferred stock}
}

\newglossaryentry{stakeholder}{
    name={stakeholder},
    description={A stakeholder is any person, legal or natural, with an economic interest in a company, including all debt, option and stock holders.},
    plural={stakeholders}
}

\newglossaryentry{share}{
    name={share},
    description={Shares represent a fraction of ownershp in a company. The number of shares a stock holder owns is the starting point for calculating their ownership stake in the company.},
    plural={shares}
}

\newglossaryentry{exercise}{
    name={exercise},
    description={Stock options are exercised and become stocks. The strike price is the price at which the stock options can be converted to equity. They can only be exercised after they have been vested.},
    plural={exercises}
}

\newglossaryentry{staged-financing}{
    name={staged financing},
    description={Staged financing is a financing strategy in which a company raises funds in stages. The first stage is typically called the seed round, with subsequent stages receiving a latin alphabet letter (such as Series A, Series B, etc.). Staged financing allows investors to reduce their risk by investing in stages, and allows the company to raise funds as it grows.},
    plural={staged financing}
}

\newglossaryentry{financing-round}{
    name={financing round},
    description={Each stage of financing is also called a financing round. Each financing round is typically led by a lead investor, who sets the terms of the financing round. The terms of the financing round include the valuation of the company, the price per share, and the type of security issued.},
    plural={financing rounds}
}
\newglossaryentry{sweat-equity}{
    name={sweat equity},
    description={Startup founders raise capital by selling shares of their stock to investors. For example, a investor might take 20 of 100 shares, leaving founder with 80 shares that were not issued against cash. This complement of the shares that held by investors is called sweat equity.},
    plural={sweat equity}
}

\newglossaryentry{dilution}{
    name={dilution},
    description={Dilution is the reduction in ownership stake that occurs when new shares are issued. In the special case that in the financing round investors purchase new shares proportionally to their then current stake, no dilution occurs. Founders and employees can also be diluted by the issuance of new shares.},
    plural={dilution}
}

\newglossaryentry{asset-class}{
    name={asset class},
    description={An asset class is a group of securities that have similar characteristics. Stocks, bonds, and real estate are all asset classes.},
    plural={asset classes}
}

\newglossaryentry{stock-class}{
    name={stock class},
    description={A stock class is a group of stocks that have similar characteristics. Common stock and preferred stock are both stock classes.},
    plural={stock classes}
}

\newglossaryentry{transaction}{
    name={transaction},
    description={A transaction refers to the issuance, change, transfer and cancellation of securities. A transaction is typically initiated by a stakeholder, and must be approved by the company. Most transactions involve a cost, with money changing hands in the opposite direction of the securities.},
    plural={transactions}
}

\newglossaryentry{equity}{
  name={equity},
  description={Equity represents ownership interest in a company. It can come in the form of stocks or stock options and may be granted to founders, employees, or investors. Equity holders have a claim to the profits of the company.},
  plural={equity}
}

\newglossaryentry{venture-capital}{
  name={venture capital},
  description={Venture capital is a form of private equity financing that is provided by venture capital firms to startups and early-stage companies that have been deemed to have high growth potential or which have demonstrated high growth},
  plural={venture capital}
}

\newglossaryentry{angel-investor}{
  name={angel investor},
  description={An angel investor is an individual who provides capital for a business start-up, usually in exchange for convertible debt or ownership equity. These investors typically support startups in the early stages of growth},
  plural={angel investors}
}

\newglossaryentry{liquidity-event}{
  name={liquidity event},
  description={A liquidity event is an event that allows a company's stakeholders to cash out or make a profit from their investment. This could be a sale of the company (also known as an exit), an IPO, or a large dividend distribution},
  plural={liquidity events}
}

\newglossaryentry{issuance}{
  name={issuance},
  description={An issuance is the creation of a new security}
  plural={issuances}
}

% An entry for IPOs
\newglossaryentry{ipo}{
  name={initial public offering},
  description={During an initial public offering, a company sells shares of its stock to the public for the first time. This is also known as "going public" and is a way for companies to raise capital for new investments or to pay off existing debt.},
  plural={initial public offerings}
}

% acquisition
\newglossaryentry{acquisition}{
  name={acquisition},
  description={An acquisition is when one company purchases another company. Depending on how many shares are acquired, the acquiring company may gain control of the acquired company.},
  plural={acquisitions}
}